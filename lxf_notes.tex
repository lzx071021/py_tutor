\documentclass{report}
\usepackage[utf-8]{inputenc}
\usepackage{hyperref}
\usepackage{amsmath}

\title{lxf Python Tutorial Note}
\author{Shicong Nie \thanks{liaoxuefeng}}
\date{\today}

\begin{document}
\begin{titlepage}
    \maketitle
\end{titlepage}

\tableofcontents

\section{简介}
适合: Web应用:网站,后台服务
      日常小工具
      包装其它语言开发的程序,方便使用
慢
代码不能加密
\section{安装python}
解释型语言,需要解释器:CPython,IPython,PyPy,etc。
\section{第一个Python程序}
I/O:print()无返回值,input()返回str。
\section{Python基础}
1. 数据类型和变量
   r' ',表示' '内部的字符串默认不转义
   '''...''',表示多行内容
   布尔运算:and,or,not
   空值None
   变量数据类型不固定 --> 动态语言
   没有常量
2. 字符编码
   ASCII --> 英文 + 符号
   Unicode --> 所有语言 + 符号
   简化版utf-8
   内存使用Unicode,硬盘使用utf-8
   Python的字符串使用Unicode,ord(),chr()
   str类型,用于展示 $\xleftrightarrow[decode()]{encode()}$ bytes类型,用于传输和保存
   格式化:\%d, \%f, \%s, \%2d, \%05d, \%.2f
   str.format()
3. list和tuple
   list,可变有序序列:.append(), .insert(), .pop()
   tuple,(“指向”)不可变有序序列,元素在定义时就确定
   单个元素的tuple:t = (x,)
4. 条件判断
   if x 为 True <-- x为非零数、非空str、非空list、tuple、etc
   if else, elif
5. 循环
   for a in b, while
   break, continue
6. 使用dict和set
   dict:key-value,x in d, d.get(),d.pop()
   key为不可变对象
   set:自动去重复,s.add(),s.remove() 
   dict和set的原理相同,唯一区别在于set没有存储value
\section{函数}
1. 调用函数
   函数名是函数对象的引用,可赋值给其他变量
2. 定义函数
   空函数pass
   参数检查,isinstance()
   返回“多个值” --> 一个tuple
   没有return --> return None 
3. 函数的参数
   顺序:位置(必选)、默认、可变、关键字、命名关键字
   默认参数必须指向不变对象;利用None 
   可变,*
   关键字,**
   命名关键字,*分割
   任意函数的调用:func(*args, **kw)
4. 递归函数
   栈溢出 --> 尾递归优化:始终占用一个栈帧;每次迭代的结果传入递归函数中
   尾递归 = 循环:函数返回自身函数对象,即return语句不能包含表达式
   Python未对尾递归做优化
\section{高级特性}
1. 切片(slicing):[start:step:stop], list, tuple, str 
2. 迭代(iteration):for --> 可迭代对象
   判断可迭代:from collections import Iterable
             isinstance(x, Iterable)
   dict默认迭代key,要迭代value --> for v in d.values()
   同时迭代key,value --> for k, v in d.items() 
   enumerate \Rightarrow 索引-元素对
3. 列表生成式(list comprehension)
   list(range(...))
   [x for if] 
4. 生成器(generator)
   生成器定义的是算法,按需使用,当要生成的东西数量不确定或很大时使用
   (),next(g) --> StopIteration, for x in g --> No StopIteration 
   修改函数得到生成器:yield
   捕获生成器函数return语句的返回值:StopIteration $\xrightarrow{try except}$ e.value 
5. 迭代器(Iterator)
   可被next()调用并不断返回下一个值的对象
   Iterator != Iterable 
   判断:isinstance(x, Iterator) 
   惰性序列,按需计算(惰性计算):调用next()时才计算,for = next 
   Iterable: list, dict, str, etc. 
   Iterator: generator, iter(Iterable) 
\section{函数式编程}
函数式:函数本身可作为参数,允许返回函数
纯函数式:没有变量,输入确定时输出确定,没有副作用
Python允许变量 --> 非纯函数式
1. 高阶函数
   至少满足其一:接受函数参数;返回函数
   abs()函数调用abs函数本身,abs是指向函数的引用
    (1)map() / reduce()
        from functools import map, reduce 
        map()返回惰性序列,reduce()
        函数内定义函数
        lambda只有一句表达式,可用于简化代码
    (2)filter()
        built-in func 
        返回惰性序列
        关键在于编写筛选函数
    (3)sorted()
        sorted(Iterable, *, key, reverse)
        关键在于编写key映射函数
2. 返回函数
   闭包(Closure):惰性函数,相关参数和变量都保存在返回的函数中
   使用()才被执行
   返回闭包(被返回的函数)时,闭包不要引用任何循环变量,或者后续会发生变化的变量
3. 匿名函数lambda
4. 装饰器(decorator)
   在代码运行期间动态增加功能的方式
   本质上,是一个返回函数的高阶函数,同时接受被装饰的函数作为参数
   @log $\Leftrightarrow$ now = log(now) 
   原来的函数对象依然存在,但其函数名变量指向了新的被装饰后的函数
   decorator本身需要参数 --> 再编写一个外层的返回decorator的高阶函数(也是一个decorator)
   防止依赖函数签名的代码出错,需要把原始函数的 __name__ 等属性复制到装饰器返回的函数 --> functools模块的wraps(): @functools.wraps(func)
5. 偏函数
   functools.partial():创建一个新的函数,固定住原函数的部分参数,简化原函数参数传入,方便调用
\section{模块}
.py = module 
包(package):带有 __init__.py 的目录
__init__.py 的模块名是它所在的包的名字(命名从最外层的包开始)
1. 使用模块
   #! /usr/bin/env python3
   #! -*- coding: utf-8 -*- 
   模块注释
   __author__ = ...
   import module 
   sys.argv 
   if __name__ == '__main__' 
   作用域:public, private:命名前带 _ ,只是形式、编程习惯
2. 安装第三方模块
   运行时临时添加路径 --> import sys 
                       sys.path.append() 
   直接添加:设置环境变量PYTHONPATH
\section{面向对象编程}
OOP:把对象(Object)作为程序的基本单元,一个对象包含了数据(property / attribute)和操作数据的函数(method)
OOP程序:一组对象的集合,对象间传递消息,使程序运行
类(class),实例(instance)
三大特点:数据封装,继承,多态
1. 类和实例
   class Student(Object)
   def __init__(self, ...)
   数据封装 --> method 
   Python允许对实例动态绑定新的属性和数据,不管__init__()中有没有定义
2. 访问限制
   在属性前面加双下划线 __ --> 私有变量,但依然不能完全阻止,只是被解释器修改了属性名,e.g. __name --> _Student__name 
   get,set方法 --> 用来给外部访问和设置私有变量,方便对参数做检查
   __xx__,特殊变量
   单下划线,约定私有
3. 继承和动态
   子类subclass,父类superclass
   class Dog(Animal)
   多态:子类继承的同名方法可以被修改;父类符合即可传入:自动调用实际类型的同名方法
   调用方只管调用,不管细节
   class = 自定义的数据类型,isinstance()
   “开闭”原则:对扩展开放,对修改封闭
   Python动态语言,不对参数做类型检查 --> 鸭子类型(duck typing):像鸭子,就是鸭子。
   该对象只要实现了对应的方法,就可被调用,故Python不要求严格的继承体系
4. 获取对象信息
   type(),返回class
   import types, type(f) == types.FunctionType --> 判断f是否是函数 
   isinstance() --> 判断具有继承关系的class,同时判断指定类型及其子类。总是优先使用
   isinstance(xx, (tuple)),判断是否是tuple中所含类型的一种
   dir() --> object的所有属性和方法
   len() $\Leftrightarrow$ __len__() 
   getattr(), setattr(), hasattr() 
5. 实例属性和类属性 
   实例属性:通过self或实例变量定义,各个实例互不干扰 
   类属性:类所有,实例共用 
   避免重名:实例属性比类属性优先级高,同名类属性将被覆盖
\section{面向对象高级编程}
1. 使用__slots__ 
   Python允许给实例动态(运行时)绑定新的属性和方法
   给单个实例绑定方法:from types import MethodType
                    s.set_age = MethodType(f, s)
   所有实例 --> 绑定class: Student.set_age = f,或直接在class中定义 
   特殊变量__slots__ --> 限制实例的属性:e.g. __slots__ = ('name', 'age')
   只对当前类的实例起作用,对继承的子类不起作用 
2. 使用@property 
   @property装饰器:将方法作为属性调用
   避免使用getattr和setter方法:getter和setter变成两个同名函数 --> 方便外部操作
   e.g. @property 
        def score(...)

        @property
        def score(...)
   只读属性:只定义getter方法(通过@property装饰)
   被广泛使用,内置
3. 多重继承
   class Dog(Mammal, Runnable)
   一个子类同时获得多个父类的所有功能 
   多重继承允许MixIn:“混入”额外的功能。
   设计类时,优先考虑通过多重继承来组合多个MixIn的功能,而不是设计多层次的复杂的继承关系 
4. 定制类 
   利用特殊方法名(special method names)
   __slots__, __len__, __str__, __repr__, __iter__, __next__, __getitem__, __setitem__, __getattr__, __call__ 
   REST API 
   通常__str__()与__repr__()函数体一样,偷懒 --> __str__ = __repr__ 
5. 使用枚举类 
   from enum import Enum 
   Enum把一组相关常量定义在一个class中,该class不可变,而且成员(类属性)可以相互比较 
   可以用成员名称引用枚举常量,又可以直接根据value的值获得获得枚举常量(value是int类型,默认从1开始计数)
   更精确地控制枚举成员的类型 --> 从Enum派生(继承)出自定义类
   from enum import unique, @unique帮助检查重复值 
6. 使用元类 
   一个类的类型是type,一个类的实例的类型是这个类(的名字)
   Python的class是运行时通过type()动态创建的
   元类metaclass $\Rightarrow$ 类 $\Rightarrow$ 实例,类是类的实例 
   强大,魔术;需要创建很多个类时可以使用
\section{错误、调试和测试}
1. 错误处理 
   try except except else finally 
   Python的错误也是类,所有的错误类型都继承自BaseException 
   所以except捕获当前类型的错误及该错误类型的子类 
   try except跨越多层调用
   错误的调用栈 
   记录错误,并让程序继续执行 --> import logging
                              logging.exception(e) 
   抛出错误:一个错误是该错误类型的实例,是被有意创建并抛出的。raise ValueError(),可以单独写raise 
   尽量使用Python内置的错误类型 
   捕获 + 抛出:捕获只是记录,处理错误需要将错误抛出,让顶层调用者处理 
   except中的raise:转化错误类型。合理的转化逻辑 
   文档中应写明可能会抛出哪些错误,以及错误产生的原因 
2. 调试 
   print() 
   assert, 关闭assert(当成pass)--> python -O err.py  
   logging: import logging 
            logging.basicConfig(level=...)
            logging.info(...) 
   level有四个从低到高的级别,高级别覆盖高级别。logging可以输出到文档
   pdb:import pdb 
        python -m pdb err.py 
        命令:l, n, p, q, etc. 
        pdf.set_trace() 
   IDE 
   logging是终极武器
3. 单元测试 
   有效地测试某个程序模块的行为,方便未来重构 
   测试用例覆盖常用的输入组合、边界条件和异常;测试代码要非常简单,防止自身出bug 
   class TestDict(unittest.TestCase)
   def test_xx()
       self.assertEqual() 
   python -m unitest mydict_test 
   setUp(), tearDown() 
4. 文档测试 
   if __name__ == '__main__' 
       import doctest
       doctest.testmode()
   >>>expr 
   expected 
   文档生成工具可以自动把包含doctest的注释提取出来 
\section{I/O编程(同步I/O)}
同步、异步I/O
Input:输入内存;Output:输出内存 
1. 文件读写 
   try finally
   f.open(), f.read(), f.write(), f.close()  
   with open(xx, 'x') as f 
   read(), read(size), readline(), readlines()
   对于配置文件,readlines()最方便 
   file-like Object:有read()方法的object 
   open(..., encoding=, error=)
2. StringIO和BytesIO 
   在内存中操作str和bytes的方法,使其和读写文件具有一致的接口
3. 操作文件和目录
   os模块、os.path模块:直接调用操作系统提供的操作文件和目录的接口函数,OS dependent 
4. 序列化
   把变量从内存中变成可存储或可传输的过程 
   Python中叫pickling,其他语言叫serialization, flattening,etc
   反序列化unpickling 
   pickle模块:.dumps(), .dump(), .loads(), .load()
   序列化为标准格式,方便传递:JSON,XML,etc 
   JSON更通用,更符合Web标准
   json模块:.dumps(), etc.。通过传入函数参数定制序列化或反序列化的规则

\end{document}